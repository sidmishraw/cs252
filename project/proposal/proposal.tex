%
% proposal.tex
%
% @author Sidharth Mishra
% @description CS 252 Project Proposal
% This latex document is the project proposal for the CS 252 class project
% @created Thu Oct 05 2017 12:14:02 GMT-0700 (PDT)
% @last-modified Thu Oct 05 2017 12:14:02 GMT-0700 (PDT)
%

\documentclass{article}
\usepackage[us, 12hr]{datetime}  % us format for date and time, time is in 12hr format
% \usepackage{fancyhdr} % for headers and footers
\usepackage{array}
\usepackage{bibentry}
\usepackage{hyperref}

% for text wrapping
\newcolumntype{L}{>{\flushleft\arraybackslash}m{7cm}}

% to use roman numerals for sections and subsections
\renewcommand{\thesection}{\Roman{section}} 
\renewcommand{\thesubsection}{\thesection.\Roman{subsection}}

\setlength{\headheight}{36pt}
% \pagestyle{fancyplain}

\title{Optimistic Software Transactional Memory Prototype}

\author {
  Name: Sidharth Mishra \\
  Class: CS 252 \\
  Assignment: Project Proposal \\
  Due Date: 10/6/2017 - 11:59 PM \\
}
\date{Last Modified: {\today}:{\currenttime}}
\begin{document}
    \maketitle

    \centering{\section{Motivation}}
    \begin{flushleft}
    {
      % block for introduction
      Concurrency primitives built into languages such as \emph{{\tt Java}} are powerful but have complex syntax and require careful use. For example, Java provides the {\tt synchronized} blocks and methods for granular locks, but these require careful handling because they might lead to degraded performance due to excessive locking or data-race conditions if not applied at correct spots. Moreover, modern languages such as \emph{{\tt Go}} supply simpler concurrency directives such as {\tt goroutines} and {\tt channels} but, these too, at times, require locking and can be difficult to reason about. One possible solution is to use \emph{Software Transactional Memory(STM)} to handle concurrency. The STM relieves the developer from thinking about or writing parallel code by letting them write serial code and the STM manager handling the system-specific details to run their code in parallel.
    }  
    \end{flushleft}

    \centering{\section{Objective}}
    \begin{flushleft}
      {
        % block for objective goes here
        The objective of this project is to build a working prototype of a optimistic STM using the working principles mentioned in \cite{Shavit1997} and \cite{MichaelW1}. The transactions in the STM will be using timestamp(versioning) protocol and there will be no locks used.
        \newline
        \newline
        Language(s) of choice:
        \begin{itemize}
          \item Java 1.8.0\_144
          \item Go 1.9
        \end{itemize}
      }
    \end{flushleft}
    \centering{\section{Progression Timeline}}
    {
      % block for progression timeline table
      \newcommand{\schedule}[2]{
        \hline
        #1 & #2
        \\
        \hline
      }

      % table\
      \begin{center}
        \begin{table}[!th]
          \begin{tabular}{|L|L|}
            \schedule{Week 1: October 10 - October 16}{Read through \cite{Shavit1997} and implement the Java version}
            \schedule{Week 2: October 23 - October 29}{Look up Go, implement working prototype in Go}
            \schedule{Week 3: November 1 - November 7}{Refine implementation of Java version, implement small examples}
            \schedule{Week 4: November 9 - November 15}{Refine implementation of Go version, implement small examples}
            \schedule{Week 5: November 18 - November 24}{Work on deliverables(code and presentation files)}
          \end{tabular}
        \end{table}  
      \end{center}
    }

    \nocite{SJonesBeautifulConcc}
    % references and citations
    \bibliography{prop-refs}{}
    \bibliographystyle{IEEEtran}
\end{document}