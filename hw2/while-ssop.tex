%
% while-ssop.tex
% @author Sidharth Mishra
% @description WHILE language operational semantics using Small Step Operational Semantics
% @created Sun Sep 24 2017 11:25:55 GMT-0700 (PDT)
% @last-modified Sun Sep 24 2017 11:25:55 GMT-0700 (PDT)
%

\documentclass{article}

\usepackage{fullpage}
\usepackage{listings}
\usepackage{amsmath}
\usepackage{amsthm}
\usepackage{amssymb}
\usepackage{hyperref}
\usepackage{url}
\usepackage{float}
\usepackage{paralist}

\title{Homework 2: Operational Semantics for WHILE}

\author{
  Name: Sidharth Mishra \\
  Class: CS 252 \\
  Assigment: HW2 \\
  Due Date: 9/28/2017 11:59 PM \\
  }
\date{Sun Sep 24 2017 11:25:55 GMT-0700 (PDT)}

\begin{document}
\maketitle

\section{Operational Semantics of WHILE using evaluation order rules}

% Commands for formatting figure
\newcommand{\mydefhead}[2]{\multicolumn{2}{l}{{#1}}&\mbox{\emph{#2}}\\}
\newcommand{\mydefcase}[2]{\qquad\qquad& #1 &\mbox{#2}\\}

% Commands for language format
\newcommand{\assign}[2]{#1~{:=}~#2}
\newcommand{\ife}[3]{\mbox{\tt if}~{#1}~\mbox{\tt then}~{#2}~\mbox{\tt else}~{#3}}
\newcommand{\whilee}[2]{\mbox{\tt while}~(#1)~#2}
\newcommand{\true}{\mbox{\tt true}}
\newcommand{\false}{\mbox{\tt false}}
\newcommand{\enot}[1]{\mbox{\tt not}~{#1}}
\newcommand{\eand}[2]{{#1}~\mbox{\tt and}~{#2}}
\newcommand{\eor}[2]{{#1}~\mbox{\tt or}~{#2}}

The language specification for WHILE language and Small Step operational semantics for the language are described below.
\\
\\
% beginning the language specification
\subsection{WHILE language specification}

% llr means that the array like table has 3 columns and first 2 are left aligned and the last is right aligned
\[
  \begin{array}{llr}
    \mydefhead{e ::=\qquad\qquad\qquad\qquad}{Expressions}
    \mydefcase{x}{variables/addresses}
    \mydefcase{v}{values}
    \mydefcase{\assign x e}{assignment}
    \mydefcase{e; e}{sequential expressions}
    \mydefcase{e ~ \tt op ~ e}{binary operations}
    \mydefcase{\ife e e e}{conditional expressions}
    \mydefcase{\whilee e e}{while expressions}
    \\
    \mydefcase{\enot e}{negation/not unary operation}
    \mydefcase{\eand e e}{{\tt and} operation}
    \mydefcase{\eor e e}{{\tt or} operation}

    \\
    \\
    \mydefhead{v ::=\qquad\qquad\qquad\qquad}{Values}
    \mydefcase{i}{integer values}
    \mydefcase{b}{boolean values}

    \\
    \\
    op ::= & + ~|~ - ~|~ * ~|~ / ~|~ > ~|~ >= ~|~ < ~|~ <= & \mbox{\emph{Binary operators}} \\
  \end{array}
\]
\\
\\
% beginning of Small Step operational semantics for the WHILE language
\subsection{Small Step operational semantics for WHILE language}

Let `$\sigma$` refer to the \textbf{\emph{Store}} which holds the state of the different variables used in the expressions.

% ssrule defines the rules for evaluation of expressions
% in WHILE language using Small Step operational semantics
%
% @param #1: the name of the rule
% @param #2: the expression to be evaluated
% @param #3: the text to be displayed on the top of the line, the mid step
%       evaluation or where condition(?)
% @param #4: the value of the evaluation
\newcommand{\ssrule}[4]{
  \multicolumn{1}{r}{
    \strut\mbox{\tt [{#1}]\qquad}
  } &
  \multicolumn{1}{l}{
    \dfrac{\strut\begin{array}{@{}c@{}} #3 \end{array}}
          {\strut\begin{array}{@{}c@{}} #2 ~~ \rightarrow ~~ #4 \end{array}}
  }
  \\~\\
}

% rules page #1
\[
  \begin{array}{r@{\qquad}l}
    
    \hline

    % 1 - value evaluates to value
    \ssrule{
      SS-VALUE
    }{
      v,~\sigma
    }{
    }{
      v,~\sigma
    }

    \hline
    \\

    % 2 - variable/address access
    \ssrule{
      SS-VARIABLE-ACCESS-REDUCTION
    }{
      x,~\sigma
    }{
      x \in domain(\sigma) \qquad \sigma(x) = v
    }{
      v,~\sigma
    }

    \hline

    % 3 - assignment(s)
    \ssrule{
      SS-ASSIGNMENT-REDUCTION
    }{
      \assign{x}{v},~\sigma
    }{
    }{
      v,~\sigma[\assign{x}{v}]
    }

    \ssrule{
      SS-ASSIGNMENT-CONTEXT
    }{
      \assign{x}{e},~\sigma
    }{
      e,~\sigma ~~ \rightarrow ~~ e',~\sigma'
    }{
      \assign{x}{e'},~\sigma'
    }

    \hline

    % 4 - sequence(s)
    \ssrule{
      SS-SEQUENCE-REDUCTION
    }{
      v; e,~\sigma
    }{
    }{
      e,~\sigma
    }

    \ssrule{
      SS-SEQUENCE-CONTEXT
    }{
      e_1; e_2,~\sigma
    }{
      e_1,~\sigma~~\rightarrow~~e_1',~\sigma'
    }{
      e_1'; e_2,~\sigma'
    }

    \hline

    % 5 - operations
    \ssrule{
      SS-OPERATION-REDUCTION
    }{
      v_1 ~\textbf{op} ~v_2,~\sigma
    }{
      v~=~\textbf{apply}~\textbf{op} ~v_1~v_2
    }{
      v,~\sigma
    }

    \ssrule{
      SS-OPERATION-CONTEXT \#2
    }{
      v ~\textbf{op} ~e,~\sigma
    }{
      e,~\sigma~~\rightarrow~~e',~\sigma'
    }{
      v ~\textbf{op} ~e',~\sigma'
    }

    \ssrule{
      SS-OPERATION-CONTEXT \#1
    }{
      e_1 ~\textbf{op} ~e_2,~\sigma
    }{
      e_1,~\sigma~~\rightarrow~~e_1',~\sigma'
    }{
      e_1' ~\textbf{op} ~e_2,~\sigma'
    }

    \hline

     % 6 - if conditionals
     \ssrule{
      SS-IF-TRUE-REDUCTION
    }{
      \ife{\true}{e_1}{e_2},~\sigma
    }{
    }{
      e_1,~\sigma
    }

    \ssrule{
      SS-IF-FALSE-REDUCTION
    }{
      \ife{\false}{e_1}{e_2},~\sigma
    }{
    }{
      e_2,~\sigma
    }

    \ssrule{
      SS-IF-CONTEXT
    }{
      \ife{e_1}{e_2}{e_3},~\sigma
    }{
      e_1,~\sigma~~\rightarrow~~e_1',~\sigma'
    }{
      \ife{e_1'}{e_2}{e_3},~\sigma'
    }

    \hline

    % 7 - while
    \ssrule{
      SS-WHILE-CONTEXT
    }{
      \whilee{e1}{e2},~\sigma
    }{
    }{
      \ife{e1}{e2;\whilee{e1}{e2}}{\false},~\sigma
    }

    \hline

\end{array}
\]

  \pagebreak

  % rules page #2
\[
  \begin{array}{r@{\qquad}l}
   
    \hline

    % I'm going to represent not in terms of if-else
    % 8 - not
    \ssrule{
      SS-NOT-CONTEXT
    }{
      \enot{e},~\sigma
    }{
    }{
      \ife{e}{\false}{\true},~\sigma
    }

    % \ssrule{
    %   SS-NOT-FALSE-REDUCTION
    % }{
    %   \enot{\false},~\sigma
    % }{
    % }{
    %   \true,~\sigma
    % }

    % \ssrule{
    %   SS-NOT-CONTEXT
    % }{
    %   \enot{e},~\sigma
    % }{
    %   e,~\sigma~~\rightarrow~~e',~\sigma'
    % }{
    %   \enot{e'},~\sigma'
    % }

    \hline

    % I'm going to represent and and or in terms of if-else
    % 9 - and
    
    \ssrule{
      SS-AND-CONTEXT
    }{
      \eand{e1}{e2},~\sigma
    }{
    }{
      \ife{e1}{\ife{e2}{\true}{\false}}{\false},~\sigma
    }

    % \ssrule{
    %   SS-AND-TRUE-FALSE-REDUCTION
    % }{
    %   \eand{\true}{\false},~\sigma
    % }{
    % }{
    %   \false,~\sigma
    % }

    % \ssrule{
    %   SS-AND-FALSE-REDUCTION
    % }{
    %   \eand{\false}{e},~\sigma
    % }{
    % }{
    %   \false,~\sigma
    % }

    % \ssrule{
    %   SS-AND-TRUE-CONTEXT
    % }{
    %   \eand{\true}{e},~\sigma
    % }{
    %   e,~\sigma~~\rightarrow~~e',~\sigma'
    % }{
    %   \eand{\true}{e'},~\sigma'
    % }

    % \ssrule{
    %   SS-AND-CONTEXT
    % }{
    %   \eand{e_1}{e_2},~\sigma
    % }{
    %   e_1,~\sigma~~\rightarrow~~e_1',~\sigma'
    % }{
    %   \eand{e_1'}{e_2},~\sigma'
    % }

    \hline

    % 10 - or
    \ssrule{
      SS-OR-CONTEXT
    }{
      \eor{e1}{e2},~\sigma
    }{
    }{
      \ife{e1}{\true}{\ife{e2}{\true}{\false}},~\sigma
    }

    % \ssrule{
    %   SS-OR-FALSE-TRUE-REDUCTION
    % }{
    %   \eor{\false}{\true},~\sigma
    % }{
    % }{
    %   \true,~\sigma
    % }

    % \ssrule{
    %   SS-OR-TRUE-REDUCTION
    % }{
    %   \eor{\true}{e},~\sigma
    % }{
    % }{
    %   \true,~\sigma
    % }

    % \ssrule{
    %   SS-OR-VAL-CONTEXT
    % }{
    %   \eor{v}{e},~\sigma
    % }{
    %   e,~\sigma~~\rightarrow~~e',~\sigma'
    % }{
    %   \eor{v}{e'},~\sigma'
    % }

    % \ssrule{
    %   SS-OR-CONTEXT
    % }{
    %   \eor{e_1}{e_2},~\sigma
    % }{
    %   e_1,~\sigma~~\rightarrow~~e_1',~\sigma'
    % }{
    %   \eor{e_1'}{e_2},~\sigma'
    % }

    \hline

  \end{array}
\]

\end{document}