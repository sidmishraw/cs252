\documentclass{article}

\usepackage{fullpage}
\usepackage{listings}
\usepackage{amsmath}
\usepackage{amsthm}
\usepackage{amssymb}
%\usepackagen{url}
\usepackage{float}
\usepackage{paralist}

\floatstyle{boxed}
\restylefloat{figure}



\newcommand{\rel}[1]{ \mbox{\sc [#1]} }

\title{Homework 2: Operational Semantics for WHILE}

\author{
  CS 252: Advanced Programming Languages \\
  Prof. Thomas H. Austin \\
  San Jos\'{e} State University \\
  }
\date{}

\begin{document}
\maketitle

\section{Introduction}

For this assignment,
you will implement the semantics for a small imperative language, named WHILE.

% Commands for formatting figure
\newcommand{\mydefhead}[2]{\multicolumn{2}{l}{{#1}}&\mbox{\emph{#2}}\\}
\newcommand{\mydefcase}[2]{\qquad\qquad& #1 &\mbox{#2}\\}

% Commands for language format
\newcommand{\assign}[2]{#1~{:=}~#2}
\newcommand{\ife}[3]{\mbox{\tt if}~{#1}~\mbox{\tt then}~{#2}~\mbox{\tt else}~{#3}}
\newcommand{\whilee}[2]{\mbox{\tt while}~(#1)~#2}
\newcommand{\true}{\mbox{\tt true}}
\newcommand{\false}{\mbox{\tt false}}

\begin{figure}\label{fig:lang}
\caption{The WHILE language}
\[
\begin{array}{llr}
  \mydefhead{e ::=\qquad\qquad\qquad\qquad}{Expressions}
  \mydefcase{x}{variables/addresses}
  \mydefcase{v}{values}
  \mydefcase{\assign x e}{assignment}
  \mydefcase{e; e}{sequential expressions}
  \mydefcase{e ~op~ e}{binary operations}
  \mydefcase{\ife e e e}{conditional expressions}
  \mydefcase{\whilee e e}{while expressions}
  \\
  \mydefhead{v ::=\qquad\qquad\qquad\qquad}{Values}
  \mydefcase{i}{integer values}
  \mydefcase{b}{boolean values}
  \\
  op ::= & + ~|~ - ~|~ * ~|~ / ~|~ > ~|~ >= ~|~ < ~|~ <=  & \mbox{\emph{Binary operators}} \\
\end{array}
\]
\end{figure}

The language for WHILE is given in Figure~\ref{fig:lang}.
Unlike the Bool* language we discussed previously,
WHILE supports \emph{mutable references}.
The state of these references is maintained in a \emph{store},
a mapping of references to values.
(``Store'' can be thought of as a synonym for heap.)
Once we have mutable references, other language constructs become more useful,
such as sequencing operations ($e_1;e_2$).



%---------
\section{Small-step semantics}

\newcommand{\ssrule}[3]{
  \rel{#1} &
  \frac{\strut\begin{array}{@{}c@{}} #2 \end{array}}
       {\strut\begin{array}{@{}c@{}} #3 \end{array}}
   \\~\\
}
\newcommand{\sstep}[4]{\ctxt[{#1}],{#2} \rightarrow \ctxt[{#3}],{#4}}
\newcommand{\sstepraw}[4]{{#1},{#2} \rightarrow {#3},{#4}}
\newcommand{\ctxt}{C}

The small-step semantics for WHILE are given in Figure~\ref{fig:smallstep}.
For the sake of brevity, these rules use \emph{evaluation contexts} ($\ctxt$),
which specify which \emph{redex} will be evaluated next.
The evaluation rules then apply to the ``hole'' ($\bullet$) in this context.

Most of these rules are fairly straightforward, but there are a couple of points
to note with the $\rel{ss-while}$ rule.
First of all, this is the only rule that makes a more complex expression
when it has finished.
%This rule is much cleaner when specified with the big-step operational semantics.

Secondly, note the final value of this expression once the while loop completes.
It will \emph{always} be {\false} when it completes.
We could have created a special value, such as {\tt null},
or we could have made the while loop a statement that returns no value.
Both choices, however, would complicate our language needlessly.


%--------------
\section{YOUR ASSIGNMENT}
\newcommand{\bstep}[4]{{#1},{#2} \Downarrow {#3},{#4}}

\noindent
{\bf Part 1:}
Rewrite the operational semantic rules for WHILE in \LaTeX\
to remove the contexts (\ctxt[...])
and to use evaluation order rules instead.
Submit both your \LaTeX\ source and the generated PDF file.

Extend your semantics with features to handle boolean values.
Specifically, add support for:
\begin{compactitem}
  \item {\tt and}
  \item {\tt or}
  \item {\tt not}
\end{compactitem}

The exact behavior of these new features is up to you,
but should seem reasonable to most programmers.

\bigskip
\noindent
{\bf Part 2:}
Once you have your semantics defined,
download {\tt WhileInterp.hs} and implement the {\tt evaluate} function,
as well as any additional functions you need.
Your implementation must be consistent with your operational semantics,
{\it including your extensions for {\tt and}, {\tt or}, and {\tt not}}.
Also, you may not change any type signatures provided in the file.

Finally, implement the interpreter to match your semantics.

\bigskip
\noindent
{\bf Zip all files together into {\tt hw2.zip} and submit to Canvas.}





\begin{figure}[H]\label{fig:smallstep}
\caption{Small-step semantics for WHILE}
{\bf Runtime Syntax:}
\[
\begin{array}{rclcl}
  \ctxt & \in & {Context} \quad & ::= & \quad \ctxt; e
        ~|~ \ctxt ~op~ e
        ~|~ v ~op~ \ctxt
        ~|~ \assign{x}{\ctxt}
        ~|~ \ife{\ctxt}{e_1}{e_2}
        ~|~ \bullet \\
  \sigma & \in & {Store} \quad  & = & \quad {variable} ~\rightarrow ~v \\
  \\
\end{array}
\]
{\bf Evaluation Rules:~~~ \fbox{$\sstepraw{e}{\sigma}{e'}{\sigma'}$}} \\
\[
\begin{array}{r@{\qquad\qquad}l}
\ssrule{ss-var}{
  x \in domain(\sigma) \qquad \sigma(x)=v
}{
  \sstep{x}{\sigma}{v}{\sigma}
}
\ssrule{ss-assign}{
}{
  \sstep{\assign{x}{v}}{\sigma}{v}{\sigma[x:=v]}
}
\ssrule{ss-op}{
  v = v_1 ~op~ v_2
}{
  \sstep{v_1~op~v_2}{\sigma}{v}{\sigma}
}
\ssrule{ss-seq}{
}{
  \sstep{v;e}{\sigma}{e}{\sigma}
}
\ssrule{ss-iftrue}{
}{
  \sstep{\ife{\true}{e_1}{e_2}}{\sigma}{e_1}{\sigma}
}
\ssrule{ss-iffalse}{
}{
  \sstep{\ife{\false}{e_1}{e_2}}{\sigma}{e_2}{\sigma}
}
\ssrule{ss-while}{
}{
  \sstep{\whilee{e_1}{e_2}}{\sigma}{\ife{e_1}{e_2;\whilee{e_1}{e_2}}{\false}}{\sigma}
}
\end{array}
\]
\end{figure}



\end{document}

